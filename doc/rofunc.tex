\documentclass[letterpaper, 10 pt, conference]{IEEEtran}
\onecolumn
% \documentclass[letterpaper, 10 pt, conference]{ieeeconf}
\IEEEoverridecommandlockouts
% The preceding line is only needed to identify funding in the first footnote. If that is unneeded, please comment it out.
% \overrideIEEEmargins                                      
% Needed to meet printer requirements.
%\usepackage{vmargin}
\usepackage{geometry}
\geometry{left=1.69cm,right=1.69cm,top=2.01cm,bottom=1.6cm} 
% \usepackage{ciaite}
\usepackage{amsmath,amssymb,amsfonts, easyReview}
\usepackage{graphicx}
\usepackage{textcomp}
\usepackage{multirow}
\usepackage{xcolor}
\usepackage{algorithm}  
\usepackage{booktabs}
\usepackage{algorithm,algpseudocode}
\usepackage{soul,color}
\usepackage{bm}
\algnewcommand{\algorithmicforeach}{\textbf{for each}}
\algdef{SE}[FOR]{ForEach}{EndForEach}[1]
{\algorithmicforeach\ #1\ \algorithmicdo}% \ForEach{#1}
{\algorithmicend\ \algorithmicforeach}% \EndForEach
\usepackage{amsmath}
\usepackage{threeparttable}
\usepackage{subfigure}
%\usepackage[colorlinks,linkcolor=blue]{hyperref}%
\usepackage{url}
%\setpapersize{USletter}
\renewcommand{\algorithmicrequire}{\textbf{Input:}}  % Use Input in the format of Algorithm  
\renewcommand{\algorithmicensure}{\textbf{Output:}} % Use Output in the format of Algorithm  
\newtheorem{df}{\bf Definition}[section]
\newtheorem{as}{\bf Assumption}[section]
\newtheorem{theorem}{\bf Theorem}[section]
\newtheorem{lm}{\bf Lemma}[section]
\newenvironment{proof}{\quad{\noindent\it Proof.}\quad}{\hfill $\square$\par}
\usepackage{amsmath}
\usepackage{amsfonts}
\usepackage{amssymb}
\usepackage{algorithm,algpseudocode}
\usepackage{hyperref}
\makeatletter
\newcommand{\algmargin}{\the\ALG@thistlm}
\usepackage{soul}
\soulregister\cite7 % 针对\cite命令
\soulregister\citep7 % 针对\citep命令
\soulregister\citet7 % 针对\citet命令
\soulregister\ref7 % 针对\ref命令
\soulregister\pageref7 % 针对\pageref命令
\makeatother
\newlength{\whilewidth}
\settowidth{\whilewidth}{\algorithmicwhile\ }
\algdef{SE}[parWHILE]{parWhile}{EndparWhile}[1]
{\parbox[t]{\dimexpr\linewidth-\algmargin}{%
		\hangindent\whilewidth\strut\algorithmicwhile\ #1\ \algorithmicdo\strut}}{\algorithmicend\ \algorithmicwhile}%
\algnewcommand{\parState}[1]{\State%
	\parbox[t]{\dimexpr\linewidth-\algmargin}{\strut #1\strut}}


\def\BibTeX{{\rm B\kern-.05em{\sc i\kern-.025em b}\kern-.08em
		T\kern-.1667em\lower.7ex\hbox{E}\kern-.125emX}}

\title{\LARGE \bf
	Rofunc: Useful Functions Package for Robot Experiments
} 

\author{Junjia Liu$^{1}$, \IEEEmembership{Student Member, IEEE}  and Fei Chen$^{\dagger 1}$, \IEEEmembership{Senior Member, IEEE}% <-this % stops a space
	\thanks{This work was supported in part by the Research Grants Council of the Hong Kong Special Administrative Region, China (Ref. No. 24209021), the VC Fund 4930745 of the CUHK T Stone Robotics Institute and CUHK Direct Grant for Research 4055140.}% <-this % stops a space
	
	\thanks{$^{1}$Junjia Liu and Fei Chen are with the Department of Mechanical and Automation Engineering, T-Stone Robotics Institute, The Chinese University of Hong Kong, Hong Kong (e-mail: {\tt\small jjliu@mae.cuhk.edu.hk, f.chen@ieee.org}).}%
	
	\thanks{$^{2}$Sylvain Calinon is with the Idiap Research Institute, Martigny, Switzerland  (e-mail: {\tt\small sylvain.calinon@idiap.ch}).}%
	\thanks{$^\dagger$Corresponding authors}
}

\begin{document}
	\maketitle
	\thispagestyle{empty}
	\pagestyle{empty}
	
	\begin{abstract}   
		This letter describes an approach to achieve well-known Chinese cooking art stir-fry on a bimanual robot system. Stir-fry requires a sequence of highly dynamic coordinated movements, which is usually difficult to learn for a chef, let alone transfer to robots. In this letter, we define a canonical stir-fry movement, and then propose a decoupled framework for learning this deformable object manipulation from human demonstration. First, dual arms of the robot are decoupled into different roles (a leader and follower) and learned with classical and neural network based methods separately, then the bimanual task is transformed into a coordination problem. To obtain general bimanual coordination, we secondly propose a Graph and Transformer based model --- \textit{Structured-Transformer}, to capture the spatio-temporal relationship between dual-arm movements. Finally, by adding visual feedback of contents deformation, our framework can adjust the movements automatically to achieve the desired stir-fry effect. We verify the framework by a simulator and deploy it on a real bimanual Panda robot system. The experimental results validate our framework can realize the bimanual robot stir-fry motion and have the potential to extend to other deformable objects with bimanual coordination.
	\end{abstract}
	
	\begin{IEEEkeywords}
		Non-prehensile manipulation, bimanual manipulation,
		spatio-temporal relationship, stir-fry, robot cooking
	\end{IEEEkeywords}
	
	\section{Introduction}

%	\begin{figure}[htp]
%		\centering
%		\includegraphics[width=0.75\linewidth]{./img/Ral_new2.png}
%		\caption{Robot stir-fry is a non-prehensile manipulation of semi-fluid objects which requires highly dynamic movements and continuous bimanual coordination in a long time series.}  
%		\label{robotstirfry}
%	\end{figure}
	
	
%	\bibliographystyle{ieeetr}
%	\bibliography{ref_rf}
	
\end{document}